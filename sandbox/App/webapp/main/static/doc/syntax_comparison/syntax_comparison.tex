\documentclass{article}
%\documentclass[prl,amsmath,amssymb]{revtex4} % PRL

% margins of 1 inch:
\setlength{\topmargin}{-.5in}
\setlength{\textheight}{9in}
\setlength{\oddsidemargin}{0in}
\setlength{\textwidth}{6.5in}

\usepackage[pdftex]{hyperref} % hyperlink equation and bibliographic citations
\usepackage[dvips]{graphicx,color}
\usepackage{amsmath} % advanced math
\usepackage{verbatim}
\usepackage{natbib} % bibilography 
\usepackage{mciteplus} % collapse multiple citations in bibilography
\usepackage{multicol}
\usepackage[toc,page]{appendix} % http://tex.stackexchange.com/questions/49643/making-appendix-for-thesis


% from http://www.flakery.org/search/show/569
%\newcommand{\infint}{\ensuremath{\int_{-\infty}^{\infty}}}
\newcommand{\cf}{\textit{c.f.}} % "compare". In context the abbreviation advises readers to consult other material, drawing attention to related ideas that provide additional arguments or information.
\newcommand{\ie}{\textit{i.e.}} % i.e. is used to explain, clarify or rephrase a statement
\newcommand{\eg}{\textit{e.g.}} % “for the sake of example”. Used to introduce an example or list of examples to illustrate what is being discussed.
\newcommand{\eqn}[1]{Eq.\ (\ref{#1})}
\newcommand{\pfrac}[2]{\ensuremath{\frac{\partial #1}{\partial #2}}}
\newcommand{\pdg}{Physics Derivation Graph}

\begin{document}
\title{Comparison of Syntax Formats for the Physics Derivation Graph}

\author{Ben Payne$^{1}$\footnote{Corresponding author: ben.is.located@gmail.com}, Michael Goff$^{2}$\\
{\it $^{1}$Department of Fun, University Name \& Town, city, State Zip}\\
{\it $^{2}$Department of Mathematics, University of Maryland, Baltimore 21228}}

\date{\today}


\maketitle % declares end of title page

\begin{abstract}
A comparison of candidates for how to express elements of the graph. It is assumed here that the Introduction to the \pdg\ has been read. 
\end{abstract}

\begin{multicols}{2}

%\tableofcontents

%\newpage

% Syntax: scope = ?

% Syntax comparison: Input, display, into, out of 

\section{Introduction}

% relevant project background
The \pdg\ is a project designed to capture mathematical physics knowledge. 

% Other overviews:
% http://tex.stackexchange.com/questions/57717/relationship-between-mathml-and-tex
% http://www.intmath.com/blog/mathematics/comparison-math-web-publishing-options-9915

% what are we comparing?
In this report, we compare methods of capturing mathematical syntax required to describe derivations in physics. This survey covers \LaTeX, Mathematical Markup Language (MathML)\cite{2014_MathML}, Mathematica\cite{2014_mathematica}, and SymPy\cite{2014_SymPy}. For MathML, both Presentation and Content forms are included.

User cares about previous experience, how wide spread in their community, speed, ease of input, presentation (rendering), ability to access content across devices, OS independence, ease of setup. 

Evaluate criteria relevant to users, including the ability to manually input syntax (section \ref{sec:quant_compare}), the ability to transform between representations (section \ref{sec:transform}), and the ability to audit correctness (section \ref{sec:audit_correctness}).

It is vital that a single syntax be used for the graph content. Suppose each syntax is used for its intended purpose -- \LaTeX for rendering equation, SymPy for the CAS, and MathML for portability. This introduces a significant source of error when a single equation requires three distinct representation. The manual entry could result in the three representations not being sychronized. Thus, a single representation satisfying multiple criteria is needed. If no single syntax meets all the needs of the Physics Derivation Graph, then the requirements must be prioritized.

This comparison is between syntax methods which do not serve the same purpose. \LaTeX is a type-setting language, while Mathematica and SymPy are Computer Algebra Systems (CAS). The reason these approaches for rendering and CAS were picked is twofold: they are widely used in the scientific community and they address requirements for the Physics Derivation Graph. 

We can ignore syntax methods which do not support notation necessary for describing physics. Example of this include ASCII\cite{1968_ASCII} and HTML\cite{1999_HTML}. Storage of the generated content (essentially a knowledge base for all of physics) isn't expected to exceed a Gigabyte, so compactness in terms of storage isn't a criterion in this evaluation.

\section{Test Cases\label{sec:test_cases}}
In order to demonstrate the variety of uses in distinct domains of Physics, a set of test cases are provided in this section. These cases are not meant to be exhaustive of either the syntax or the scientific domain. Rather, they are examples of both capability requirements of the Physics Derivation Graph and of the syntax methods. 

% https://www.physicsforums.com/threads/formulae-of-various-topics.102736/

Case 1 is a second order polynomial. Algebra
\begin{equation}
a x^2 + b x + c = 0
\label{eq:polynomial_case1_body}
\end{equation}

Case 2, Stoke's theorem, includes integrals, cross products, and vectors. Calculus
\begin{equation}
\int \int_{\sum} \vec{\nabla} \times \vec{F} \dot d\sum = \oint_{\partial \sum} \vec{F}\dot d\vec{r}
\label{eq:stokes_case2_body}
\end{equation}

Case 3: Tensor analysis. Einstein notation: contravariant = superscript, covariant = subscript. Used in electrodynamics
\begin{equation}
Y^i(X_j) = \delta^i_{\ j}
\label{eq:tensor_analysis_case3_body}
\end{equation}

Case 4 covers notation used in Quantum Mechanics. Symbols such as $\hbar$ and Dirac notation are typically used.

Case 4a is the creation operator 
\begin{equation}
\hat{a}^+ |n\rangle = \sqrt{n+1} |n+1\rangle
\label{eq:creation_operator_case4a_body}
\end{equation}

Case 4b is the uncertainty principle
\begin{equation}
\sigma_x \sigma_p \geq \frac{\hbar}{2}
\label{eq:uncertainty_principle_case4b_body}
\end{equation}

Case 4c: L\"{u}ders projection
\begin{equation}
 |\psi\rangle \rightarrow \sum_n  |c_n|^2 P_n,\ \rm{where}\ P_n = \sum_i |\psi_{ni}\rangle \langle \psi_{ni}|
\label{eq:Luders_projection_case4c_body}
\end{equation}

\section{Quantitative Comparison of Test Cases\label{sec:quant_compare}}

\end{multicols}
\begin{center}
\begin{table}
\caption {Character Count of Test Cases} \label{table:char_count} 
\begin{center}
\begin{tabular}{|c|c|c|c|c|c|c|}\hline
Name            & case 1 & case 2 & case 3 & case 4a & case 4b & case 4c \\\hline
Latex            & 20       & 101     & 26       & 45         &   39      & 110 \\\hline
PMathML      &  324     & 538    & 348     & 372       &  250      &        \\\hline
CMathML      &   381    &           &            &             &              &        \\\hline
Mathematica &             &           &            &             &              &        \\\hline
SymPy           &             &           &            &             &              &        \\\hline
\end{tabular}
\end{center}
\end{table}
\end{center}
\begin{multicols}{2}

Character count for the MathML was carried out using
\begin{verbatim}
wc -m mathML_presentation_case*.xml
\end{verbatim}

\section{Qualitative Comparisons of Syntax Methods\label{sec:qual_compare}}
\LaTeX, MathML, and SymPy are free and open source. Mathematica is proprietary and not free.

For Physicists comfortable writing journal articles in \LaTeX or exploring ideas in Mathematica, these are natural syntax methods. Both \LaTeX and Mathematica are concise, making them intuitive to read and quick to enter. MathML is a verbose syntax which is lengthy to manually enter and yield difficult to read the native XML.  

Unicode is needed to support Dirac notation and any other non-ASCII text in MathML

\subsection{Transform Syntax\label{sec:transform}}

Wolfram Research offers the ability to convert from Mathematica expressions to MathML on their site \href{http://www.mathmlcentral.com/Tools/ToMathML.jsp}{www.mathmlcentral.com}

A CAS typically produces output syntax such as \LaTeX or MathML in a single format. However, there are often many ways to represent the same math,  \eg~Eq.~\ref{eq:example_partial_derivative_representations}.

\subsection{Audit Correctness of Derivations\label{sec:audit_correctness}}

One reason Computer Algebra Systems such as Mathematica and SymPy were included in this survey was to address the requirement of checking correctness of derivations. 

\LaTeX and Presentation MathML are intended for rendering equations and are not easily parsed consistently by a CAS. For example, scientists and mathematicians often render the same partial differential operation in multiple ways:
\begin{equation}
\frac{\partial^2}{\partial t^2}F =\frac{\partial}{\partial t}\frac{\partial F}{\partial t} = \frac{\partial^2 F}{\partial t^2} = \frac{\partial}{\partial t}\dot{F} = \frac{\partial \dot{F}}{\partial t} = \ddot{F}.
\label{eq:example_partial_derivative_representations}
\end{equation}
All of these are equivalent. 

\section{Conclusions}

This survey covered \LaTeX, Mathematica, MathML, and SymPy as candidates for syntax methods to be used for the Physics Derivation Graph. Although \LaTeX is intuitive for scientists and is concise, it is a typesetting language and not well suited for the web or use in Computer Algebra Systems (CAS). Mathematica is also concise and has wide spread use by scientists, though its cost limits accessibility. Mathematica is also proprietary, which limits the ability to explore the correctness of this CAS

\LaTeX is concise and is widely used in the scientific community. It does not work well for portability to other representations and is ill-suited for use by CAS. For the initial phases of development for the Physics Derivation Graph, portability and compatibilty with a CAS are not the priority. Since getting content into the graph is the priority, the \LaTeX representation will be used.

If a different syntax is needed in the future (\ie, for a CAS), then each \LaTeX expression will need to be translated.

\section{Bibliography}

\bibliographystyle{unsrt}
\bibliography{../bibliography} % external bibtex flat-file database
\end{multicols}

\newpage
\appendix
%\begin{appendices}
\section{Test cases in Latex and MathML}\label{sec:appendix_test_cases}

\subsection{Case 1: polynomial}

\begin{equation}
a x^2 + b x + c = 0
\label{eq:polynomial_case1}
\end{equation}
Latex: 
\begin{verbatim}
a x^2 + b x + c = 0
\end{verbatim}

SymPy:
\verbatiminput{sympy_case1_polynomial.py}

Presentation MathML:
% http://www.mathmlcentral.com/Tools/FromMathML.jsp

%\begin{figure}
%\begin{center}
%\includegraphics[scale=1,bb=0 0 111 19]{images/case1_polynomial_mathML_presentation.gif}
%\caption{Case 1 polynomial in Presentation MathML}
%\end{center}
%\end{figure}

\verbatiminput{mathML_presentation_case1_polynomial.xml}

Content MathML:
\verbatiminput{mathML_content_case1_polynomial.xml}

\subsection{Case 2: Stoke's theorem}
\begin{equation}
\int \int_{\sum} \vec{\nabla} \times \vec{F} \dot d\sum = \oint_{\partial \sum} \vec{F}\dot d\vec{r}
\label{eq:stokes_case2}
\end{equation}
Latex:
\begin{verbatim}
\int \int_{\sum} \vec{\nabla} \times \vec{F} \dot d\sum = 
\oint_{\partial \sum} \vec{F}\dot d\vec{r}
\end{verbatim}

SymPy:
\verbatiminput{sympy_case2_stokes.py}


Presentation MathML:
\verbatiminput{mathML_presentation_case2_stokes.xml}

Content MathML:
\begin{verbatim}
<math xmlns="http://www.w3.org/1998/Math/MathML">

</math>
\end{verbatim}

\subsection{Case 3: Tensor analysis}
\begin{equation}
Y^i(X_j) = \delta^i_{\ j}
\label{eq:tensor_analysis_case3}
\end{equation}
Latex: 
\begin{verbatim}
Y^i(X_j) = \delta^i_{\ j}
\end{verbatim}

SymPy:
\verbatiminput{sympy_case3_tensor.py}


Presentation MathML:
\verbatiminput{mathML_presentation_case3_tensor.xml}

Content MathML:
\begin{verbatim}
<math xmlns="http://www.w3.org/1998/Math/MathML">

</math>
\end{verbatim}


\subsection{Case 4a: creation operator}
\begin{equation}
\hat{a}^+ |n\rangle = \sqrt{n+1} |n+1\rangle
\label{eq:creation_operator_case4a}
\end{equation}

\begin{verbatim}
\hat{a}^+ |n\rangle = \sqrt{n+1} |n+1\rangle
\end{verbatim}

SymPy:
\verbatiminput{sympy_case4a_creation.py}


Presentation MathML:
\verbatiminput{mathML_presentation_case4a_creation.xml}

Content MathML:
\begin{verbatim}
<math xmlns="http://www.w3.org/1998/Math/MathML">

</math>
\end{verbatim}

\subsection{Case 4b: uncertainty principle}
\begin{equation}
\sigma_x \sigma_p \geq \frac{\hbar}{2}
\label{eq:uncertainty_principle_case4b}
\end{equation}

\begin{verbatim}
\sigma_x \sigma_p \geq \frac{\hbar}{2}
\end{verbatim}

SymPy:
\verbatiminput{sympy_case4b_uncertainty.py}


Presentation MathML:
\verbatiminput{mathML_presentation_case4b_uncertainty.xml}

Content MathML:
\begin{verbatim}
<math xmlns="http://www.w3.org/1998/Math/MathML">

</math>
\end{verbatim}

\subsection{Case 4c: L\"{u}ders projection}
\begin{equation}
 |\psi\rangle \rightarrow \sum_n  |c_n|^2 P_n,\ \rm{where}\ P_n = 
 \sum_i |\psi_{ni}\rangle \langle \psi_{ni}|
\label{eq:Luders_projection_case4c}
\end{equation}

\begin{verbatim}
 |\psi\rangle \rightarrow \sum_n  |c_n|^2 P_n,\ \rm{where}\ P_n = \sum_i |\psi_{ni}\rangle \langle \psi_{ni}|
\end{verbatim}

SymPy:
\verbatiminput{sympy_case4c_projection.py}


Presentation MathML:
\begin{verbatim}
<math xmlns="http://www.w3.org/1998/Math/MathML">

</math>
\end{verbatim}

Content MathML:
\begin{verbatim}
<math xmlns="http://www.w3.org/1998/Math/MathML">

</math>
\end{verbatim}

%\end{appendices}


\end{document}
