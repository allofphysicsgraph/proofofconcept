\documentclass{article}
%\documentclass[prl,amsmath,amssymb]{revtex4} % PRL

% margins of 1 inch:
\setlength{\topmargin}{-.5in}
\setlength{\textheight}{9in}
\setlength{\oddsidemargin}{0in}
\setlength{\textwidth}{6.5in}

\usepackage[pdftex]{hyperref} % hyperlink equation and bibliographic citations
\usepackage[dvips]{graphicx,color}
\usepackage{amsmath} % advanced math
\usepackage{verbatim}
\usepackage{natbib} % bibilography 
\usepackage{mciteplus} % collapse multiple citations in bibilography
\usepackage{multicol}
\usepackage[toc,page]{appendix} % http://tex.stackexchange.com/questions/49643/making-appendix-for-thesis


% from http://www.flakery.org/search/show/569
%\newcommand{\infint}{\ensuremath{\int_{-\infty}^{\infty}}}
\newcommand{\cf}{\textit{c.f.}} % "compare". In context the abbreviation advises readers to consult other material, drawing attention to related ideas that provide additional arguments or information.
\newcommand{\ie}{\textit{i.e.}} % i.e. is used to explain, clarify or rephrase a statement
\newcommand{\eg}{\textit{e.g.}} % “for the sake of example”. Used to introduce an example or list of examples to illustrate what is being discussed.
\newcommand{\eqn}[1]{Eq.\ (\ref{#1})}
\newcommand{\pfrac}[2]{\ensuremath{\frac{\partial #1}{\partial #2}}}
\newcommand{\pdg}{Physics Derivation Graph}

\begin{document}
\title{Comparison of Databases for the \pdg}

\author{Ben Payne$^{1}$\footnote{Corresponding author: ben.is.located@gmail.com}, Michael Goff$^{2}$\\
{\it $^{1}$Department of Fun, University Name \& Town, city, State Zip}\\
{\it $^{2}$Department of Mathematics, University of Maryland \& Baltimore 21228}}

\date{\today}


\maketitle % declares end of title page

\begin{abstract}
A comparison of candidates for how to store elements of the graph. 
It is assumed here that the Introduction to the \pdg has been read. 
\end{abstract}

\begin{multicols}{2}

\section{Introduction}

% relevant project background
The \pdg\ is a project designed to capture mathematical physics knowledge. 

% what are we comparing?

In this article, we compare candidate data format options for storage of content for the \pdg. 
Candidate formats include XML\cite{2008_XML}, comma-separated plain-text (CSV), JSON key-value pairs, standard textbooks, Wikipedia, proprietary formats (i.e. Wolfram Alpha, Symbolab, FormulaDatabase), and other websites on the Internet (i.e. HyperPhysics). 
\end{multicols}
\begin{tabular}{|l|c|c|c|c|c|c|c|c|}\hline
                     & XML & CSV & key-value & JSON & textbooks     & Wikipedia & proprietary   & generic websites \\\hline
free               & yes  & yes   & yes           & yes     & generally no & yes           & generally no & generally yes \\\hline
open source & yes   & yes  & yes            & yes    & generally no & yes            & generally no & generally yes \\\hline
\end{tabular}
\begin{multicols}{2}

\section{Conclusions}
The \pdg\ uses a dictionary with nested dictionaries and lists. This data structure is suitable to JSON.

\section{Bibliography}

\bibliographystyle{unsrt}
\bibliography{../bibliography} % external bibtex flat-file database
\end{multicols}

%\newpage
%\appendix
%\section{Test cases in Latex and MathML}\label{sec:appendix_test_cases}

\subsection{Case 1: polynomial}

\begin{equation}
a x^2 + b x + c = 0
\label{eq:polynomial_case1}
\end{equation}
Latex: 
\begin{verbatim}
a x^2 + b x + c = 0
\end{verbatim}

SymPy:
\verbatiminput{sympy_case1_polynomial.py}

Presentation MathML:
% http://www.mathmlcentral.com/Tools/FromMathML.jsp

%\begin{figure}
%\begin{center}
%\includegraphics[scale=1,bb=0 0 111 19]{images/case1_polynomial_mathML_presentation.gif}
%\caption{Case 1 polynomial in Presentation MathML}
%\end{center}
%\end{figure}

\verbatiminput{mathML_presentation_case1_polynomial.xml}

Content MathML:
\verbatiminput{mathML_content_case1_polynomial.xml}

\subsection{Case 2: Stoke's theorem}
\begin{equation}
\int \int_{\sum} \vec{\nabla} \times \vec{F} \dot d\sum = \oint_{\partial \sum} \vec{F}\dot d\vec{r}
\label{eq:stokes_case2}
\end{equation}
Latex:
\begin{verbatim}
\int \int_{\sum} \vec{\nabla} \times \vec{F} \dot d\sum = 
\oint_{\partial \sum} \vec{F}\dot d\vec{r}
\end{verbatim}

SymPy:
\verbatiminput{sympy_case2_stokes.py}


Presentation MathML:
\verbatiminput{mathML_presentation_case2_stokes.xml}

Content MathML:
\begin{verbatim}
<math xmlns="http://www.w3.org/1998/Math/MathML">

</math>
\end{verbatim}

\subsection{Case 3: Tensor analysis}
\begin{equation}
Y^i(X_j) = \delta^i_{\ j}
\label{eq:tensor_analysis_case3}
\end{equation}
Latex: 
\begin{verbatim}
Y^i(X_j) = \delta^i_{\ j}
\end{verbatim}

SymPy:
\verbatiminput{sympy_case3_tensor.py}


Presentation MathML:
\verbatiminput{mathML_presentation_case3_tensor.xml}

Content MathML:
\begin{verbatim}
<math xmlns="http://www.w3.org/1998/Math/MathML">

</math>
\end{verbatim}


\subsection{Case 4a: creation operator}
\begin{equation}
\hat{a}^+ |n\rangle = \sqrt{n+1} |n+1\rangle
\label{eq:creation_operator_case4a}
\end{equation}

\begin{verbatim}
\hat{a}^+ |n\rangle = \sqrt{n+1} |n+1\rangle
\end{verbatim}

SymPy:
\verbatiminput{sympy_case4a_creation.py}


Presentation MathML:
\verbatiminput{mathML_presentation_case4a_creation.xml}

Content MathML:
\begin{verbatim}
<math xmlns="http://www.w3.org/1998/Math/MathML">

</math>
\end{verbatim}

\subsection{Case 4b: uncertainty principle}
\begin{equation}
\sigma_x \sigma_p \geq \frac{\hbar}{2}
\label{eq:uncertainty_principle_case4b}
\end{equation}

\begin{verbatim}
\sigma_x \sigma_p \geq \frac{\hbar}{2}
\end{verbatim}

SymPy:
\verbatiminput{sympy_case4b_uncertainty.py}


Presentation MathML:
\verbatiminput{mathML_presentation_case4b_uncertainty.xml}

Content MathML:
\begin{verbatim}
<math xmlns="http://www.w3.org/1998/Math/MathML">

</math>
\end{verbatim}

\subsection{Case 4c: L\"{u}ders projection}
\begin{equation}
 |\psi\rangle \rightarrow \sum_n  |c_n|^2 P_n,\ \rm{where}\ P_n = 
 \sum_i |\psi_{ni}\rangle \langle \psi_{ni}|
\label{eq:Luders_projection_case4c}
\end{equation}

\begin{verbatim}
 |\psi\rangle \rightarrow \sum_n  |c_n|^2 P_n,\ \rm{where}\ P_n = \sum_i |\psi_{ni}\rangle \langle \psi_{ni}|
\end{verbatim}

SymPy:
\verbatiminput{sympy_case4c_projection.py}


Presentation MathML:
\begin{verbatim}
<math xmlns="http://www.w3.org/1998/Math/MathML">

</math>
\end{verbatim}

Content MathML:
\begin{verbatim}
<math xmlns="http://www.w3.org/1998/Math/MathML">

</math>
\end{verbatim}



\end{document}
