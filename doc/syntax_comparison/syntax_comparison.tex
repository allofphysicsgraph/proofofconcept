\documentclass{article}
%\documentclass[prl,amsmath,amssymb]{revtex4} % PRL

% margins of 1 inch:
\setlength{\topmargin}{-.5in}
\setlength{\textheight}{9.5in}
\setlength{\oddsidemargin}{0in}
\setlength{\textwidth}{6.5in}

\usepackage[pdftex]{hyperref} % hyperlink equation and bibliographic citations
\usepackage[dvips]{graphicx,color}
\usepackage{amsmath} % advanced math
\usepackage{verbatim} % multi-line comments
\usepackage{natbib} % bibilography 
\usepackage{mciteplus} % collapse multiple citations in bibilography
\usepackage{multicol}


% from http://www.flakery.org/search/show/569
%\newcommand{\infint}{\ensuremath{\int_{-\infty}^{\infty}}}
\newcommand{\ie}{\textit{i.e.}\ }
\newcommand{\eg}{\textit{e.g.}\ }
\newcommand{\eqn}[1]{Eq.\ (\ref{#1})}
\newcommand{\pfrac}[2]{\ensuremath{\frac{\partial #1}{\partial #2}}}

\begin{document}
\title{Comparison of Syntax formats}

\author{Ben Payne$^{1}$\footnote{Corresponding author: ben.is.located@gmail.com}\\
{\it $^{1}$Department of Fun, University Name \& Town, city, State Zip}}

\date{\today}

%\begin{abstract}
%blah blah blah
%\end{abstract}

\maketitle % declares end of title page
\begin{multicols}{2}

%\tableofcontents

%\newpage

\section{Introduction}

% relevant project background
The Physics derivation graph is an effort to 

Other overviews:
% http://tex.stackexchange.com/questions/57717/relationship-between-mathml-and-tex
% http://www.intmath.com/blog/mathematics/comparison-math-web-publishing-options-9915

% what are we comparing?
In this report, we compare methods of capturing mathematical syntax. 
presentation mathml, content mathml, mathematica, latex, abstract syntax tree
what is the evaluation?

can it describe all the notation necessary for physics
how easy is the input method? (character count)
can it be transformed to other representations
In general, we need to evaluate

ability to input
ability to transform between representations
ability to audit correctness
Not a problem:

storage
consumer cares about:

speed
ease of use
presentation
accessibility - multi-device, multi-OS
easy setup
To start, evaluate:

character count
rendering

Mathematical Markup Language (MathML)\cite{2014_MathML}, specifically Content MathML and Presentation MathML.

Mathematica\cite{2014_mathematica}


\section{Test Cases}
A set of test cases for flexing the capability of syntax, provided in Latex

\subsection{Case 1: polynomial}

\begin{equation}
a x^2 + b x + c = 0
\label{eq:polynomial_case1}
\end{equation}
Latex: 
\begin{verbatim}
a x^2 + b x + c = 0
\end{verbatim}

Presentation MathML:
\begin{verbatim}
<mrow>
 <mrow>
  <msup>
   <mrow><mi> x </mi></mrow>
   <mrow><mi> n </mi></mrow>
  </msup>
  <mo>+</mo>
  <msup>
   <mrow><mi> y </mi></mrow>
   <mrow><mi> n </mi></mrow>
  </msup>
 </mrow>
 <mo>=</mo>
 <msup>
  <mrow><mi> z </mi></mrow>
  <mrow><mi> n </mi></mrow>
 </msup>
</mrow>
\end{verbatim}

Content MathML:
\begin{verbatim}
<math xmlns=
"http://www.w3.org/1998/Math/MathML">
<apply>
 <eq/>
  <apply>
   <plus/>
    <apply>
     <ci>a</ci> 
     <power/><ci> x </ci><ci> 2 </ci>
     </apply>
    <apply>
     <ci>b</ci> <ci> x </ci>                    
     </apply>
    <apply>
    <ci>c</ci>                                 
    </apply>
   </apply>
   <apply>
    <cn>0</cn>
  </apply>
</apply>
</math>
\end{verbatim}

\subsection{Case 2: Stoke's theorem}
\begin{equation}
\int \int_{\sum} \vec{\nabla} \times \vec{F} \dot d\sum = \oint_{\partial \sum} \vec{F}\dot d\vec{r}
\label{eq:stokes_case2}
\end{equation}
Latex:
\begin{verbatim}
\int \int_{\sum} \vec{\nabla} \times
 \vec{F} \dot d\sum = 
\oint_{\partial \sum} \vec{F}\dot d\vec{r}
\end{verbatim}

\subsection{Case 3: Tensor analysis}
\begin{equation}
Y^i(X_j) = \Delta^i_{\ j}
\end{equation}
Latex: 
\begin{verbatim}
Y^i(X_j) = \Delta^i_{\ j}
\end{verbatim}

Einstein notation: contravariant = superscript, covariant = subscript

\subsection{Case 4a: creation operator}
\begin{equation}
\hat{a}^+ |n\rangle = \sqrt{n+1} |n+1\rangle
\end{equation}

\begin{verbatim}
\hat{a}^+ |n\rangle = \sqrt{n+1} |n+1\rangle
\end{verbatim}

\subsection{Case 4b: uncertainty principle}
\begin{equation}
\sigma_x \sigma_p \geq \frac{\hbar}{2}
\end{equation}

\begin{verbatim}
\sigma_x \sigma_p \geq \frac{\hbar}{2}
\end{verbatim}

\section{Comparison of Test Cases}

\begin{tabular}{|c|c|c|c|}\hline
Name & Cost & Open Source & keystroke count \\\hline
Latex & Free & Yes & 4 \\\hline
\end{tabular}

\section{Bibliography}

\bibliographystyle{apsrev4-1} % use the apsrevM.bst for collapsible citations
\bibliography{../bibliography} % external bibtex flat-file database

\end{multicols}
\end{document}
