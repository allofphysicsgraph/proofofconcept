\section{Test cases in Latex and MathML}\label{sec:appendix_test_cases}

\subsection{Case 1: polynomial}

\begin{equation}
a x^2 + b x + c = 0
\label{eq:polynomial_case1}
\end{equation}
Latex: 
\begin{verbatim}
a x^2 + b x + c = 0
\end{verbatim}

Presentation MathML:
% http://www.mathmlcentral.com/Tools/FromMathML.jsp

%\begin{figure}
%\begin{center}
%\includegraphics[scale=1,bb=0 0 111 19]{images/case1_polynomial_mathML_presentation.gif}
%\caption{Case 1 polynomial in Presentation MathML}
%\end{center}
%\end{figure}

\verbatiminput{mathML_presentation_case1_polynomial.xml}

Content MathML:
\verbatiminput{mathML_content_case1_polynomial.xml}

\subsection{Case 2: Stoke's theorem}
\begin{equation}
\int \int_{\sum} \vec{\nabla} \times \vec{F} \dot d\sum = \oint_{\partial \sum} \vec{F}\dot d\vec{r}
\label{eq:stokes_case2}
\end{equation}
Latex:
\begin{verbatim}
\int \int_{\sum} \vec{\nabla} \times \vec{F} \dot d\sum = 
\oint_{\partial \sum} \vec{F}\dot d\vec{r}
\end{verbatim}

Presentation MathML:
\verbatiminput{mathML_presentation_case2_stokes.xml}

Content MathML:
\begin{verbatim}
<math xmlns="http://www.w3.org/1998/Math/MathML">

</math>
\end{verbatim}

\subsection{Case 3: Tensor analysis}
\begin{equation}
Y^i(X_j) = \Delta^i_{\ j}
\label{eq:tensor_analysis_case3}
\end{equation}
Latex: 
\begin{verbatim}
Y^i(X_j) = \Delta^i_{\ j}
\end{verbatim}

Presentation MathML:
\verbatiminput{mathML_presentation_case3_tensor.xml}

Content MathML:
\begin{verbatim}
<math xmlns="http://www.w3.org/1998/Math/MathML">

</math>
\end{verbatim}


\subsection{Case 4a: creation operator}
\begin{equation}
\hat{a}^+ |n\rangle = \sqrt{n+1} |n+1\rangle
\label{eq:creation_operator_case4a}
\end{equation}

\begin{verbatim}
\hat{a}^+ |n\rangle = \sqrt{n+1} |n+1\rangle
\end{verbatim}

Presentation MathML:
\verbatiminput{mathML_presentation_case4a_creation.xml}

Content MathML:
\begin{verbatim}
<math xmlns="http://www.w3.org/1998/Math/MathML">

</math>
\end{verbatim}

\subsection{Case 4b: uncertainty principle}
\begin{equation}
\sigma_x \sigma_p \geq \frac{\hbar}{2}
\label{eq:uncertainty_principle_case4b}
\end{equation}

\begin{verbatim}
\sigma_x \sigma_p \geq \frac{\hbar}{2}
\end{verbatim}

Presentation MathML:
\verbatiminput{mathML_presentation_case4b_uncertainty.xml}

Content MathML:
\begin{verbatim}
<math xmlns="http://www.w3.org/1998/Math/MathML">

</math>
\end{verbatim}

\subsection{Case 4c: L\"{u}ders projection}
\begin{equation}
 |\psi\rangle \rightarrow \sum_n  |c_n|^2 P_n,\ \rm{where}\ P_n = 
 \sum_i |\psi_{ni}\rangle \langle \psi_{ni}|
\label{eq:Luders_projection_case4c}
\end{equation}

\begin{verbatim}
 |\psi\rangle \rightarrow \sum_n  |c_n|^2 P_n,\ \rm{where}\ P_n = \sum_i |\psi_{ni}\rangle \langle \psi_{ni}|
\end{verbatim}

Presentation MathML:
\begin{verbatim}
<math xmlns="http://www.w3.org/1998/Math/MathML">

</math>
\end{verbatim}

Content MathML:
\begin{verbatim}
<math xmlns="http://www.w3.org/1998/Math/MathML">

</math>
\end{verbatim}
